\documentclass[a4paper,twocolumn,oneside]{memoir}

\usepackage[british]{babel}
\usepackage{gurps}
\usepackage{luacode}
\luadirect{require("main.lua")}
\usepackage{libertine}
\usepackage{microtype}
\usepackage{xparse}
\usepackage{multicol}
\usepackage{graphicx}
\usepackage{tikz}
\usetikzlibrary{calc,intersections,angles,positioning}
\usepackage{tcolorbox}
\usepackage{transparent}
\usepackage{enumitem}
\usepackage{multicol}

\renewcommand{\familydefault}{\sfdefault}

% \RenewDocumentEnvironment{charactertraitlist}{}{%
%   \begin{itemize}[noitemsep]%
%   }{%
%   \end{itemize}%
% }


\tikzset{
  galaxy line/.style={
    to path={
      let \p{start}=(\tikztostart), \p{target}=(\tikztotarget),
      \p{inter}=({\x{start} + sign(\x{target}-\x{start}) * abs(\y{target}-\y{start})}, \y{target})
      in -- (\p{inter}) -- (\tikztotarget)
    },
  },
}
\tikzset{
  pics/carc/.style args={#1:#2:#3}{
    code={
      \draw[pic actions] (#1:#3) arc(#1:#2:#3);
    }
  }
}

\graphicspath{{images/}}

\setulmarginsandblock{1.5cm}{1.5cm}{*}
\setlrmarginsandblock{1.5cm}{1.5cm}{*}
\checkandfixthelayout

\setsecnumdepth{subsubsection}

% Useful macros
\NewDocumentCommand{\system}{m}{\textsc{System#1}}
\NewDocumentCommand{\player}{m}{\textbf{\mbox{#1}}}
\NewDocumentCommand{\thecompany}{}{\textbf{The Company}\xspace}

\NewDocumentCommand{\traititem}{smmm}{%
  \IfBooleanTF{#1}{             %starred
  \item \textbf{#2}[#3] p.~#4%
  }{                            %unstarred
  \item #2[#3] p.~#4%
  }%
}
\NewDocumentEnvironment{traitlist}{O{List of traits.}oO{}}%
{\begin{figure*}
    \begin{minipage}{1.0\linewidth}
      \raggedright
      \begin{multicols}{3}
        \begin{itemize}[noitemsep]}%
        {\end{itemize}
      \end{multicols}
      #3
    \end{minipage}
    \caption{#1}
    \IfValueTF{#2}{\label{#2}}{}
  \end{figure*}}


\newcounter{proposals}
\NewDocumentCommand{\listproposalsname}{}{List of proposals}
\newlistof{listofproposals}{pro}{\listproposalsname}
\NewDocumentEnvironment{proposal}{m}{%
  \refstepcounter{proposals}
  \addcontentsline{pro}{proposals}{#1}
  \noindent
  \textbf{Proposal \theproposals: #1}\hrulefill\par
}{%
  \par\noindent%
  \textbf{End of proposal}\hrulefill%
}


\makeatletter%
\NewDocumentCommand{\InnateAttackPointsPerLevel}{m}{%
  \luadirect{ia.setPointsPerLevel(_INNATE_ATTACK[_INNATE_ATTACK_KEY], #1)}%
  #1~points/level\xspace%
}
\NewDocumentCommand{\InnateAttackLevel}{m}{%
  \luadirect{ia.setLevel(_INNATE_ATTACK[_INNATE_ATTACK_KEY], #1)}%
  level~#1\xspace%
}

\NewDocumentCommand{\InnateAttackPercentModifier}{O{}m}{%
  \luadirect{
    pcmod = ia.createPercentModifier(#2/100, [[#1]])
    ia.insertPercentModifier(_INNATE_ATTACK[_INNATE_ATTACK_KEY], pcmod)
  }%
  #2\%\xspace%
}

\NewDocumentCommand{\InnateAttackFormatPercentModifier}{som}{%
  \IfBooleanTF{#1}{
    \pgfmathparse{int(round(#3*100))}\pgfmathresult\%\IfValueTF{#2}{~#2}{}%
  }{
    #3\%\IfValueTF{#2}{~#2}{}%
  }%
}

\NewDocumentCommand{\InnateAttackRemovePercentModifier}{m}{%
  \footnote{Currently have:
    \luadirect{for i,v in ipairs(_INNATE_ATTACK[_INNATE_ATTACK_KEY].percentModifiers) do
      tex.sprint(v.reason .. ",") end} but trying to remove #1.
  }%
  \luadirect{
    ia.removePercentModifier(_INNATE_ATTACK[_INNATE_ATTACK_KEY], [[#1]])
  }%
  \footnote{Now have:
    \luadirect{for i,v in ipairs(_INNATE_ATTACK[_INNATE_ATTACK_KEY].percentModifiers) do
      tex.sprint(v.reason .. ",") end} and tried to remove #1.
  }%
  % #2\%\xspace%
}

\NewDocumentCommand{\gurps@ia@totalpoints}{o}{%
  \IfValueTF{#1}{\luadirect{
      tex.sprint(tostring(ia.getTotalPoints(_INNATE_ATTACK[ [[#1]] ])))
    }}{\luadirect{
      tex.sprint(tostring(ia.getTotalPoints(_INNATE_ATTACK[_INNATE_ATTACK_KEY])))
    }}
}

\NewDocumentCommand{\InnateAttackTotalPoints}{}{%
  \gurps@ia@totalpoints~points\xspace%
}


\NewDocumentCommand{\gurps@pc}{}{\%\xspace}

% Takes a real percentage and converts it to LaTeX (e.g. 0.1 -> +10%)
% TODO make the above true!
\NewDocumentCommand{\gurps@format@pcmod}{m}{\luadirect{
    if #1 > 0 then
      tex.sprint("+" .. math.floor(#1*100))
    else
      tex.sprint(math.floor(#1*100))
    end
  }\gurps@pc}


\NewDocumentCommand{\InnateAttackBreakdown}{}{
  \begin{tabular}{lc}
    \hline
    Reason & Modifier \\\hline
    \luaexec{
      currentIA = _INNATE_ATTACK[_INNATE_ATTACK_KEY]
      tex.sprint("(Level &" .. currentIA.level .. [[)\\]])
      tex.sprint("Points/level & ×" .. currentIA.pointsPerLevel .. [[\\]])
      for i,j in ipairs(currentIA.percentModifiers) do
      tex.sprint(j.reason .. [[& \string\InnateAttackFormatPercentModifier*{]] .. j.value .. [[} \\]])
      end
    }\hline
    Total & \(\lceil\)\luadirect{tex.sprint(ia.getAdjustedPointsPerLevel(_INNATE_ATTACK[_INNATE_ATTACK_KEY]))}%
            ×%
            \luadirect{tex.sprint(_INNATE_ATTACK[_INNATE_ATTACK_KEY].level)}\(\rceil\)%
            = \InnateAttackTotalPoints \\\hline
  \end{tabular}
}

\NewDocumentCommand{\InnateAttackBasedOn}{m}{%
  \luadirect{
    _INNATE_ATTACK[_INNATE_ATTACK_KEY] = deepcopy(_INNATE_ATTACK[ [[#1]] ])
  }%
  based on \mbox{#1}\footnote{but now called `\luadirect{tex.sprint(_INNATE_ATTACK_KEY)}'}\xspace%
}
\makeatother%

\luadirect{_INNATE_ATTACK = {}}
\NewDocumentEnvironment{innateattack}{O{_}}{%
  \luadirect{
    _INNATE_ATTACK_KEY = [[#1]]
    _INNATE_ATTACK[_INNATE_ATTACK_KEY] = {
      level=0,
      pointsPerLevel=0,
      percentModifiers={}
    }}%
  \begin{itemize}}{\end{itemize}}

\NewDocumentCommand{\fakefootnotemark}{m}{\textsuperscript{#1}}
\NewDocumentCommand{\fakefootnotetext}{mm}{{\textsuperscript{#1}~{\footnotesize #2}}}


\newcommand\dodec[1]{{90+(30*(#1-1)}}

\NewDocumentCommand{\one}{}{%
  \draw[white] (0,0) circle (0pt) pic[black]{carc=\dodec{2}:\dodec{7}:6pt};%
  \draw[white] (0,0) circle (0pt) pic[black,line width={6pt*0.2}]{carc=\dodec{2}:\dodec{4}:{6pt*0.9}};%
}
\NewDocumentCommand{\two}{}{%
  \draw[white] (0,0) circle (0pt) pic[black]{carc=\dodec{7}:\dodec{2+12}:{6pt*0.5}};
  \draw[white] (0,0) circle (0pt) pic[black,line width={6pt*0.2}]{carc=\dodec{7}:\dodec{10}:{6pt*0.4}};
}

\NewDocumentCommand{\three}{}{%
  \draw[white] (0,0) circle (0pt) pic[black]{carc=\dodec{3}:\dodec{11}:6pt};
  \draw[white] (0,0) circle (0pt) pic[black,line width={6pt*0.2}]{carc=\dodec{3}:\dodec{6}:{6pt*0.9}};
  \draw[white] (0,0) circle (0pt) pic[black,line width={6pt*0.2}]{carc=\dodec{8}:\dodec{11}:{6pt*0.9}};
}

\NewDocumentCommand{\four}{}{
  \draw[white] (0,0) circle (0pt) pic[black]{carc=\dodec{8}:\dodec{6+12}:6pt};
  \draw[white] (0,0) circle (0pt) pic[black,line width={6pt*0.2}]{carc=\dodec{0}:\dodec{2}:{6pt*0.9}};
}
\NewDocumentCommand{\five}{}{
  \draw[black] (0,0) circle ({6pt*0.5});
}
\NewDocumentCommand{\ten}{}{
  \draw[white] (0,0) circle (0pt) pic[black]{carc=\dodec{2}:\dodec{8}:6pt};
  \draw[black] (0,0) -- ++(\dodec{2}:6pt);
}
\NewDocumentCommand{\fifteen}{}{
  \draw[white] (0,0) circle (0pt) pic[black]{carc=0:180:6pt};
  \draw[black] ({6pt*0.3},0) -- ++(0:{6pt*0.7});
}

\newcommand\n[1]{\begin{tikzpicture}[anchor=center]\draw[white] (0,0) circle (6pt);%\node[anchor=south east] at (1.2cm,-1.2cm) {\i};
\luadirect{print_tesseracont(totesseracont(#1))}\end{tikzpicture}}
% \NewDocumentCommand{\tesseracont}{m}{%
%   \begin{tikzpicture}%
%     \luadirect{print_tesseracont(totesseracont(#1))}%
%   \end{tikzpicture}%
% }

\title{\system{51} source book}
\author{Nathanael~Farley, David~Sumner, et al.}

%%% Local Variables:
%%% mode: latex
%%% TeX-master: "main"
%%% End:
